\PassOptionsToPackage{unicode=true}{hyperref} % options for packages loaded elsewhere
\PassOptionsToPackage{hyphens}{url}
%
\documentclass[ignorenonframetext,]{beamer}
\usepackage{pgfpages}
\setbeamertemplate{caption}[numbered]
\setbeamertemplate{caption label separator}{: }
\setbeamercolor{caption name}{fg=normal text.fg}
\beamertemplatenavigationsymbolsempty
\usepackage{lmodern}
\usepackage{amssymb,amsmath}
\usepackage{ifxetex,ifluatex}
\usepackage{fixltx2e} % provides \textsubscript
\ifnum 0\ifxetex 1\fi\ifluatex 1\fi=0 % if pdftex
  \usepackage[T1]{fontenc}
  \usepackage[utf8]{inputenc}
  \usepackage{textcomp} % provides euro and other symbols
\else % if luatex or xelatex
  \usepackage{unicode-math}
  \defaultfontfeatures{Ligatures=TeX,Scale=MatchLowercase}
\fi
% use upquote if available, for straight quotes in verbatim environments
\IfFileExists{upquote.sty}{\usepackage{upquote}}{}
% use microtype if available
\IfFileExists{microtype.sty}{%
\usepackage[]{microtype}
\UseMicrotypeSet[protrusion]{basicmath} % disable protrusion for tt fonts
}{}
\IfFileExists{parskip.sty}{%
\usepackage{parskip}
}{% else
\setlength{\parindent}{0pt}
\setlength{\parskip}{6pt plus 2pt minus 1pt}
}
\usepackage{hyperref}
\hypersetup{
            pdftitle={An Overview of rmarkdown},
            pdfauthor={Hadrien Dykiel},
            pdfborder={0 0 0},
            breaklinks=true}
\urlstyle{same}  % don't use monospace font for urls
\newif\ifbibliography
\usepackage{graphicx,grffile}
\makeatletter
\def\maxwidth{\ifdim\Gin@nat@width>\linewidth\linewidth\else\Gin@nat@width\fi}
\def\maxheight{\ifdim\Gin@nat@height>\textheight\textheight\else\Gin@nat@height\fi}
\makeatother
% Scale images if necessary, so that they will not overflow the page
% margins by default, and it is still possible to overwrite the defaults
% using explicit options in \includegraphics[width, height, ...]{}
\setkeys{Gin}{width=\maxwidth,height=\maxheight,keepaspectratio}
% Prevent slide breaks in the middle of a paragraph:
\widowpenalties 1 10000
\raggedbottom
\setbeamertemplate{part page}{
\centering
\begin{beamercolorbox}[sep=16pt,center]{part title}
  \usebeamerfont{part title}\insertpart\par
\end{beamercolorbox}
}
\setbeamertemplate{section page}{
\centering
\begin{beamercolorbox}[sep=12pt,center]{part title}
  \usebeamerfont{section title}\insertsection\par
\end{beamercolorbox}
}
\setbeamertemplate{subsection page}{
\centering
\begin{beamercolorbox}[sep=8pt,center]{part title}
  \usebeamerfont{subsection title}\insertsubsection\par
\end{beamercolorbox}
}
\AtBeginPart{
  \frame{\partpage}
}
\AtBeginSection{
  \ifbibliography
  \else
    \frame{\sectionpage}
  \fi
}
\AtBeginSubsection{
  \frame{\subsectionpage}
}
\setlength{\emergencystretch}{3em}  % prevent overfull lines
\providecommand{\tightlist}{%
  \setlength{\itemsep}{0pt}\setlength{\parskip}{0pt}}
\setcounter{secnumdepth}{0}

% set default figure placement to htbp
\makeatletter
\def\fps@figure{htbp}
\makeatother


\title{An Overview of rmarkdown}
\author{Hadrien Dykiel}
\date{July 2nd, 2018}

\begin{document}
\frame{\titlepage}

\begin{frame}{How it works}
\protect\hypertarget{how-it-works}{}

When you run render, R Markdown feeds the .Rmd file to knitr, which
executes all of the code chunks and creates a new markdown (.md)
document which includes the code and it's output.

The markdown file generated by knitr is then processed by pandoc which
is responsible for creating the finished format.

This may sound complicated, but R Markdown makes it extremely simple by
encapsulating all of the above processing into a single render function.

\begin{figure}
\centering
\includegraphics{https://d33wubrfki0l68.cloudfront.net/61d189fd9cdf955058415d3e1b28dd60e1bd7c9b/b739c/lesson-images/rmarkdownflow.png}
\caption{Render process}
\end{figure}

\end{frame}

\begin{frame}{Why is it useful?}
\protect\hypertarget{why-is-it-useful}{}

Repdroducible analyses \(\Longrightarrow\) increased efficiency, reduced
error rate

\begin{itemize}
\tightlist
\item
  Share results with stakeholders who don't care about the code
\item
  Share results \& analysis with other data practitioners
\end{itemize}

\end{frame}

\begin{frame}{What can I build with the rmarkdown package?}
\protect\hypertarget{what-can-i-build-with-the-rmarkdown-package}{}

\begin{itemize}
\tightlist
\item
  Static reports in various formarts including
  \href{http://colorado.rstudio.com/rsc/content/1129/report.html\%22}{html},
  \href{http://colorado.rstudio.com/rsc/content/1081/parameterized-happiness.pdf}{PDF},
  Word, and Powerpoint.
\item
  Websites. The \href{https://rmarkdown.rstudio.com/}{rmarkdown website}
  itself is built using rmarkdown
\item
  \href{http://svmiller.com/rmarkdown-example.pdf}{Presentations}
\item
  \href{https://gallery.shinyapps.io/cran-gauge/}{Interactive
  dashboards}
\end{itemize}

\end{frame}

\end{document}
